\begin{frame}
    \frametitle{Levels of Control}
    \framesubtitle{Runtimes I}
    \begin{itemize}
        \item OS Level Container Management is complex.
        \item Docker is a collection of products making this happen.
        \item Control of Cgroups and Namespaces is done by Container Runtimes,
    \end{itemize}
\end{frame}

\begin{frame}
    \frametitle{Levels of Control}
    \framesubtitle{Runtimes II}
    \textbf{Low Level Runtimes}\\
    \vspace{0.5cm}
    \begin{itemize}
        \item Handle Kernel feature like \texttt{cgroups} and \texttt{namespaces}.
        \item Typically don't handle container and image management.
        \item Not very user friendly.
    \end{itemize}
    \vspace{0.5cm}
    Docker uses \texttt{runc} for this.
\end{frame}

\begin{frame}
    \frametitle{Levels of Control}
    \framesubtitle{Runtimes III}
    \textbf{High Level Runtimes}\\
    \vspace{0.5cm}
    \begin{itemize}
        \item Call low-level runtimes to use Kernel features.
        \item Manage container and image lifecycles.
        \item Usually command line based and lack advanced features.
    \end{itemize}
    \vspace{0.5cm}
    Docker uses \texttt{containerd} for this.
\end{frame}

\begin{frame}
    \frametitle{Levels of Control}
    \framesubtitle{Interfaces}
    \textbf{Interfaces}\\
    To use these runtimes more user friendly, implementations like Dockers add custom interfaces for this.
    \vspace{0.5cm}
    \begin{itemize}
        \item Docker CLI for advanced CLI usage (\texttt{docker run}).
        \item Docker Dashboard for a graphical user interface.
        \item Add features for ease of use like e.g.:
        \begin{itemize}
            \item Compose
            \item Registry management
            \item Auto pulling of images
        \end{itemize}
    \end{itemize}
    \vspace{0.5cm}
    Not necessary in production environment.
\end{frame}